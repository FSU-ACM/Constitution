\documentclass{article}
\usepackage[utf8]{inputenc}
\setlength\parindent{0pt}

\title{Constitution of the Association for Computing
Machinery Student Chapter of Florida State
University}
\date{March 2020}

\begin{document}

\maketitle
\newpage
\tableofcontents
\newpage


\section{Article I: Name}

The formal name of this student chapter shall be ``Association for
Computing Machinery Student Chapter at Florida State University''.
Colloquial names for the chapter shall include but not be limited to
``ACM at FSU'', ``ACM@FSU'', ``FSU ACM'', and ``Florida State ACM''.

\section{Article II: Purpose}

\subsection{Primary Objectives}

The Chapter is organized and will be operated exclusively for education and scientific purposes (ACM Bylaw 5, Section 2). Specific objectives are:

\begin{enumerate}
\item
  To promote an increased knowledge of and greater interest in the
  science, design, development, construction, languages, management and applications of modern computing.
\item
  To promote a greater interest in computing and its applications.
\item
  To provide a means of communication between persons interested in
  computing.
\end{enumerate}

\subsection{Local Objectives}

The Chapter shall serve the students of Florida State University and other interested persons in the Tallahassee community.\\

This Chapter will operate in conjunction with the Computer Science Department at Florida State University to provide increased access to opportunities to students of the department.

\subsection{Chapter Charter}

The Chapter is chartered by the Association for Computing Machinery.

\section{Article III: Membership}

\subsection{Membership Requirements}

\subsubsection{Membership Statement}

Per a requirement set forth in the Florida Administrative Code,
recognized student organizations shall be limited to enrolled FSU
students.


\subsubsection{Hazing}

No hazing or discrimination will be used as a condition of membership in
this organization. Information regarding hazing can be found at
http://hazing.fsu.edu.

\subsubsection{University Non-Discrimination Statement}

This organization agrees to adhere to the University non-discrimination
statement: No university student may be denied membership on the basis
of race, creed, color, sex, religion, national origin, age, disability,
genetic information, veterans' or marital status, sexual orientation,
gender identity, gender expression, or any other protected group status.

\subsection{Active Membership Status}

An active chapter member is an individual who meets the requirements of
membership (Article III, Section 1) and who has attended one event
hosted by the Chapter in the current semester. Active chapter members
are by default considered to be in ``good standing'' with the Chapter.

\subsection{Revocation of Membership}

Membership may be revoked for any member who causes harm, either
physically, mentally, or financially, to any other member of the
Chapter, the ACM, or an ACM SIG, or who otherwise violates the student
code of conduct.

\subsection{Appeal Process}

Within thirty days of receiving membership revocation, the student has
the right to an appeal before the Executive Board (Article IV, Section
3). The Executive Board will grant the appeal then schedule a hearing
date within the next fourteen days after the appeal. At this hearing all
students involved in the incident and the Executive Board must be
present.\\

The format for this hearing shall consist first of the appealing student
giving their testimony of the events that lead to membership expulsion.
This will be followed by the testimonies of all other involved parties.
Once complete, all parties will be asked to leave.\\

At this point in time a vote shall be taken among the Executive Board
with a tie breaker given by the department chair of the appealing
student. The Executive Board's decision is final and an officer of the
board will inform the involved parties of the outcome.\\

Only one appeal may be granted per incident.\\

If an officer of the Executive Board is personally involved in the appeal
process, that officer's voting powers are revoked for the length of the
procedure.

\section{Article IV: Officers}

\subsection{Elected Officers}

The elected officers of the Chapter shall be (in order of prominence): a President, a Vice President, a Secretary, a Treasurer, and a Social Chair.

\subsection{Non-elected Officers}

There shall be two categories of non-elected officers for the Chapter.

\subsubsection{President Emeritus}

At the beginning of each election cycle, the immediately preceding
president shall be offered the role of President Emeritus.\\

Should the previous year's president decline the position, the role
shall be offered to the other officer's of the previous board in order
of prominence. A non-president officer filling this position will not be known as President Emeritus, but by their previous title followed by
``Emeritus'' (e.g., ``Secretary Emeritus'').\\

If no officer accepts, the role will be left unfilled.

\subsubsection{Provisionary Officers}

The elected officers shall have the authority to establish provisionary
officer roles for the Chapter. Active chapter members may be selected
for the provisionary roles by any elected officers, and confirmed with
the unanimous consent of the elected officers.\\

The title and duties of the provisionary officer will be established at
the time of that officer's confirmation.\\

The provisionary officer and their role are only valid through the end
of their appointed term. Provisionary officer terms may not outlive the
board who created their position.

\subsection{The Executive Board}

\subsubsection{Membership}

The Executive Board shall consist of the constitutionally enumerated
elected officers, the President Emeritus (or the position's substitute
as defined in Article IV Section 2), and the Chapter's faculty sponsor.

\subsubsection{Authority}

Except in circumstances and proceedings defined by this constitution,
the Executive Board shall preside over all matters in the chapter. At
all chapter gatherings, the elected officer highest in order of
prominence (Article IV Section 1) shall have immediate authority over
event proceedings.

\subsection{Officer Duties}

The primary duties of the elected officers shall be as defined below.
This section only defines the core responsibilities of each officer.
Officer duties may be expanded upon in any Bylaws established by this
chapter.

\subsubsection{President Duties}

The President shall:

\begin{enumerate}
\item
  Represent the chapter at all meetings with external parties.
\item
  Appoint all committees and committee chairs of this chapter.
\item
  Delegate various works, projects, and responsibilities to other
  officers on the board.
\item
  Have the ability to sign financial documents.
\item
  Bear responsibility of ensuring all officers are fulfilling their
  duties, and calling to attention the inability of officers to perform
  their duties to the Executive Council.
\end{enumerate}

\subsubsection{Vice President Duties}

The Vice President shall:

\begin{enumerate}
\item
  Assume the duties of the President in the event of President's
  absence.
\item
  Assume the duties of the President that are so delegated by the
  President.
\end{enumerate}

\subsubsection{Secretary Duties}

The Secretary shall:

\begin{enumerate}
\item
  Keep minutes of all officer meetings.
\item
  Keep records of attendance for all chapter gatherings.
\item
  Arrange for another officer to assume the duties of the Secretary in
  the event of the Secretary's absence.
\end{enumerate}


\subsubsection{Treasurer Duties}

The Treasurer shall:

\begin{enumerate}
\item
  Maintain financial records of the Chapter.
\item
  Maintain any financial accounts established by the Chapter relating to
  the storage or transfer of chapter funds.
\item
  Collect and maintain records of any dues established by the Chapter.
\item
  Prepare the Chapter's Annual Financial Report for ACM Headquarters.
\item
  Prepare and present a report on the current status of the Chapter's
  funds for the Executive Board at the beginning and end of each term,
  or as requested by any other member of the Executive Board.
\end{enumerate}

\subsubsection{Social Chair Duties}

The Social Chair shall:

\begin{enumerate}
\item
  Manage and maintain all accounts and outlets used for promoting events
  and proceedings of the Chapter.
\item
  Bear primary responsibility for promoting chapter events and
  communicating chapter status to members of the Chapter.
\item
  Actively gather and present interests (e.g.~ideas for workshops,
  social events, etc.) of members of the Chapter to the Executive Board
  for consideration.
\end{enumerate}


\subsubsection{Other Duties}

The officers of the chapter collectively shall fulfill the following
duties:

\begin{enumerate}
\item
  Prepare an Annual Chapter Report for presentation to the Chapter, and
  the Board of Advisors of the FSU Computer Science Department.
\item
  Prepare the Chapter's Activity Report for submission to the ACM
  Headquarters (ACM Bylaw 6, Section 7).
\item
  Submit the Annual Financial Report to ACM Headquarters. (Financial
  Accountability Policy, ACM Bylaw 6, Section 6).
\item
  Notify the ACM headquarters of changes in the elected officers or in
  the faculty sponsor of this chapter.
\item
  Submit any proposed changes in this chapter's constitution or bylaws
  to the Chairs of the Local Activities Board and the ACM Constitution
  and Constitution Committee for approval.
\item
  Host elections for the next term's officers in accordance with the
  procedures defined in Article V.
\end{enumerate}


\subsection{Officer Eligibility}

The requirements for eligibility of officership shall be as follows:

\begin{enumerate}
\item
  Officers of the Chapter must be qualify for membership of the Chapter.
\item
  No student whose current GPA is below a 3.00 out of 4.00 shall be
  eligible for officership.
\item
  Any student who has served in an office on the Executive Board in each
  of the three years before a given election shall not be eligible for
  nomination to any role on the Executive Board, except for the office of President Emeritus (or substitute office as defined in Article IV Section 2), in the immediately following year.
\end{enumerate}

\section{Article V: Selection of Elected Officers}

\subsection{Eligibility to Vote}

Voting rights shall be available to all active members of the Chapter
who are considered in good standing (Article III, Section 2).\\

Per Florida State University guidelines, the Chapter Sponsor is
ineligible to vote.

\subsection{Nomination Process}

The nomination of officers shall occur within the period between the
announcement of an election date and the election. Nominations may occur
up until and during the casting of ballots.\\

Nominations may be submitted verbally or electronically.\\

Nominated individuals must be active members in good standing with the
Chapter (Article III, Section 2), and meet the requirements of Officer
Eligibility stated in Article IV, Section 5.


\subsection{Election Process}

The election of new officers shall occur at the Annual Election Meeting,
which shall occur during each Spring semester. The Executive Board shall
publish an announcement of the meeting's date and time at least two
weeks before said date and time.\\

The meeting shall begin at the date and time specified by the
announcement. All nominated candidates shall have the opportunity to
address those assembled. Once all candidates have had the opportunity to
speak, all eligible voters will have the opportunity to vote by
privately casting their ballot. Ballots may be cast for at least thirty
minutes after the candidates have spoken, but for no more than one hour.
No ballot shall be denied if it is submitted within the one hour mark
after the candidates have finished speaking.\\

Once all ballots have been collected, the totals are to be tallied by
the two highest ranking officers of the Executive Board who are not
running in the election. If at least two officers are unavailable, the
Chapter Sponsor shall participate in counting the totals.\\

In event of a tie, the Executive Board shall hold an internal vote to
determine the winner of the tied position. If any officer of the
Executive Board is a participant in the tied race, they shall be unable
to vote, except in the case of in which the Executive Board's internal
vote is also a tie.


\subsection{Term of Office}

Elected officers of the Chapter shall serve from the beginning of the
summer semester immediately following the election until the beginning
of the following summer semester. As such, a term of office shall be
three semesters total (one full school year).

Provisionary officers of the Chapter shall serve from the time which
they are confirmed by the Executive Board until they are either
prematurely terminated from their position by the Executive Board or
until the term of the Executive Board which appointed them expires.

The President Emeritus (or substitute role) shall serve from the time at
which they accept the nomination of their position until the term of the
Executive Board with which they served expires.

An officer may participate in the reelection process for a second term
if they meet all Officer Eligibility requirements stated in Article IV,
Section 5.


\section{Article VI: Officer Vacancies}

\subsection{Removal of Officers}

Any officer can be removed from office by a 2/3rd's majority vote by all
voting members in good standing with the organization with the approval
of the Faculty Advisor. The officer must be notified in writing at least
7 business days in advance of the total vote, and be given a chance to
address the community in his or her own defense.

\subsection{Resignation}

Officers no longer wishing to serve on the board must submit their
resignation to the President at least two (2) weeks in advance. Prior to
the officer's final day he/she shall provide all documents relating to
the organization and brief his/her replacement of current projects in
his/her care.

\subsection{Filling Vacant Positions}

Vacancies in any office other than President caused by resignation,
removal, or inability to fulfill duties shall be filled by majority vote
of the Executive Board.\\

If the office of President is so vacated, the Vice President shall
immediately assume that office, and the Executive Board shall then fill
the vacant office of Vice President.\\

Should a position on the Executive Board be left empty by either
mechanism, a general election will shall be immediately held to fill the
empty position.

\section{Article VII: Faculty Advisor}

\subsection{Advisor Duties}

The faculty advisor shall monitor the activities of the Executive Board
and the chapter as a whole as to provide guidance to the Executive
Board.\\

The faculty advisor shall have no voting powers within this
organization.

\subsection{Nomination, Removal, and Replacement}

The faculty advisor shall be any full-time faculty employed by FSU with
an interest in helping this organization succeed and flourish.\\

The faculty advisor shall be nominated and approved by the Chair of the
Computer Science Department at Florida State University.\\

The faculty advisor shall serve until they resign or are removed by the
Chair of the Computer Science Department at Florida State University.

\section{Article VIII: Finances}

\subsection{Dues}

The Executive Board may elect to establish and collect dues for a period
limited to the duration of their term. The Executive Board shall
determine the benefits granted to due-paying members, but shall still
operate to the benefit of non-paying members per Article II, Section II.

No university student may be denied membership due to inability to pay
dues. If a member is not able to pay dues, other arrangements will be
made.

\subsection{Management}

This chapter shall establish a treasury at a bank selected by the
Executive Board. All chapter funds shall reside in the treasury.

The President and the Treasurer of this chapter shall have direct access
to the treasury. This includes physical access (such as a bank card) and
online access (such as credentials to an online portal).

Access to other officers of this chapter may be granted by majority vote
of the Executive Board.

Chapter funds may be spent on any expense deemed necessary by the
Executive Board.

\subsection{Dissolution}

In the event that this chapter dissolves, the final Executive Board
shall yield all assets including the treasury to the Computer Science
department at Florida State University.

\section{Article IX: Publications}


\subsection{Compliance}

All advertisements of the organization must comply with the University
Posting Policy (http://www.posting.fsu.edu/).

\subsection{Approval}

All publications including fliers, newsletters, t-shirts, or other media
shall be approved by the President of the chapter before distribution.

\section{Article X: Definitions}

\subsection{Meetings and Events}

A meeting or an event of this chapter consists of any planned gathering
of chapter members deemed as official by the Executive Board.

This chapter may only hold meetings in venues available to all members
of the Association for Computing Machinery (ACM Bylaw 12, Section 1) and
all students, faculty, or staff of Florida State University.

The Executives Board reserves the right to establish private meetings so
long as the topics of discussion are sensitive in nature.

\subsection{Good Standing}

A chapter member is considered in good standing if they meeting the
requirements for active chapter membership per Article III, Section 1
and Section 2.

A chapter member may have their good standing revoked by the Executive
Board with the approval of the Faculty Advisor. This is unique from the
revocation of chapter membership.

A chapter member shall not stay out of good standing for more than
thirty (30) days. Unless his/her membership is revoked per Article III,
Section 3, they will default back to being in good standing at the end
of the aforementioned period.

\section{Article XI: Amendments}

\subsection{Preamble}

Amendments to this constitution shall be made in accordance with the
procedures set forth in the active constitution.

\subsection{Process}

The submission and approval of amendments to this constitution shall
proceed as such: 
\begin{enumerate}
\item
    Any voting member (Article V, Section 1) may submit a proposed amendment to the Executive Council or the Faculty Advisor. 
\item
    The Secretary of the Executive Board will prepare a revised edition of the constitution with the amendment's changes incorporated. They shall also prepare a rider clearly illustrating the changes to the constitution.
\item
    A majority vote of the Executive Board may approve the changes. Should a majority vote not be achieved, the amendment process terminates.
\item
    A 2/3rds vote of eligible voting members (Article V, Section 1) may approve the new constitution. This vote must be announced two (2) weeks in advance, with the draft of the new constitution and accompanying rider made available at the time of announcement. Should fewer than ten (10) non-officer members attend, the vote shall be rescheduled (up to a maximum of 3 times, otherwise the vote is considered a failure). Should the vote fail, the amendment process terminates.
\item
    Upon success of the vote in step 4, the revised constitution is effective immediately.
\end{enumerate}

\newpage

\section{Article XII: Document History}

\begin{itemize}
\item
  1st Draft: 26th day of September 2005
\item
  1st Revision: 30th day of September 2005
\item
  2nd Draft: 2nd day of October 2005
\item
  3rd Draft: 22nd day of October 2005
\item
  2nd Revision: 13th day of April 2015
\item
  4th Draft: 18th day of September, 2017
\end{itemize}


\end{document}
